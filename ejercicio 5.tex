\documentclass[12pt,a4paper,openany]{article}
\usepackage[utf8]{inputenc}
\usepackage{amsmath}
\usepackage{amsfonts}
\usepackage{amssymb}
\usepackage{graphicx}
\usepackage{enumerate}
\usepackage{multirow} 
\usepackage[width=0.00cm, left=3.00cm, right=2.50cm, top=3.00cm, bottom=2.50cm]{geometry}
\begin{document}
	\begin{enumerate}[5]
		\item la demanda diaria de un profucto puede ser: 0,1,2,3,4 con probabilidades respectivas: 0.3 0.3 0.2 0.1 0.1.
		\begin{enumerate}
			\item describa el modelo de probabilidad de la demanda promedio de 36 dias
			\item ¿que probabilidad hay de que la demanda promedio de 36 dias este entre 1 y 2 inclusive?
		\end{enumerate}
		sol:
		\begin{enumerate}
			\item 
			\begin{table}[htbp]
				\begin{center}
					\begin{tabular}{|l|l|l|}
						\hline
						x & f(x) & xf(x)\\
						\hline \hline \hline
						0 & 0.3 & 0.0 \\ \hline
						1 & 0.3 & 0.3 \\ \hline
						2 & 0.2 & 0.4 \\ \hline
						3 & 0.1 & 0.3 \\ \hline
						4 & 0.1 & 0.4 \\ \hline
					\end{tabular}
					\caption{tabla 1.}
					\label{tabla:sencilla}
				\end{center}
			\end{table}
			
			\[
			\sum xf(x)=1.4 
			\]
			
			\[
			\sum x^2f(x)-u^2=sqrt(2.44)
			\]
			respuesta a:
			N(1.4,sqrt(2.44)/36)
			\item 
			P(1 \le  x \le 2) = P( x \le 2)-P( x \le 1)\\
			= P( x-u/0.26 \le 2-1.4)-P( x-u \le 1-1.4/0.26)\\
			=0.98928 - 0.06178
		\end{enumerate}
	\end{enumerate}
	

\end{document}