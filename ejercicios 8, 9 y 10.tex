\documentclass{article}
\usepackage{lmodern}
\usepackage{mathtools}
\usepackage[T1]{fontenc}
\usepackage[spanish,activeacute]{babel}
\providecommand{\abs}[1]{\lvert#1\rvert}
\raggedright
\usepackage[dvips]{graphicx} % LaTeX
\usepackage[usenames]{color}
\usepackage{pdfpages}

% ejercicios 8, 9 y 10 zamudio castro, antony - 27170110

\begin{document}

\textcolor{red}{8)} La vida 'util (en miles de horas) de una bater'ia es una variable aleatoria $X$ con funci'on de densidad:

$f(x)= \left\{ \begin{array}{lcc}
             2-2x,  & 0 \leq x \leq 1 \\
             \\ 0, & $en el resto$ \\
             \end{array}
   \right.$
\vspace{3mm}

Si $\bar{X}_{36}$ es la media de la muestra aleatoria $X_{1}, X_{2}, ..., X_{36}$ escogida de $X$, ¿con qu'e probabilidad $\bar{X}_{36}$ es mayor que $420$ horas?
\vspace{3mm}

\textcolor{red}{Soluci'on:}
\vspace{3mm}

$\Longrightarrow E(x) = \int_0^1 x(2-2x)dx = \int_0^1 (2x-2x^{2})dx = (\frac{2x^{2}}{2}-\frac{2x^{3}}{3}) /_0^{1} = 1-\frac{2}{3} = \frac{1}{3}$
\vspace{4mm}

$\Longrightarrow Var(x) = E(x^{2}) - E(x)^{2} = \int_0^1 x^{2}(2-2x)dx-\frac{1}{9} = \int_0^1 (2x^{2}-2x^{3})dx-\frac{1}{9} $
\vspace{2mm}

$= (\frac{2x^{3}}{3}-\frac{2x^{4}}{4}) /_0^{1} - \frac{1}{9}= \frac{2}{3}-\frac{1}{2} -\frac{1}{9}= \frac{1}{18}$
\vspace{4mm}

$\Longrightarrow P(\bar{X}>0.42) =  P(\frac{\bar{X}-\mu}{\sigma/\sqrt{n}}>\frac{0.42-1/3}{1/\sqrt{18}\sqrt{36}}) = P(Z>2.21) = 1- P(Z<2.21)$
\vspace{3mm}

$= 1-0.98645 = \fcolorbox{blue}{white}{0.01355}$
\vspace{10mm}

\textcolor{red}{9)} Sea $\bar{X}_{40}$ la media de la muestra aleatoria $X_{1}, X_{2}, ..., X_{40}$ de tamaño $n = 40$ escogida de una poblaci'on $X$ cuya distribuci'on es geom'etrica con funci'on de probabilidad:

$f(x) = \frac{1}{5}(\frac{4}{5})^{x-1}, x = 1,2,...$
\vspace{3mm}

Halle la probabilidad de que la media muestral difiera de la media poblacional en a lo m'as el $10\%$ del valor de la varianza de la poblaci'on.
\vspace{4mm}

\textcolor{red}{Soluci'on:}
\vspace{3mm}

$\Longrightarrow E(x) = \sum_{x = 1}^{\infty} xpq^{x-1} = p\sum_{x = 1}^{\infty} \frac{\partial}{\partial q} (q^{x}) = p\frac{\partial}{\partial q} (\sum_{x = 1}^{\infty} q^{x}) = p\frac{\partial}{\partial q} (\frac{q}{1-q})$ 
\vspace{2mm}

$= p\frac{1}{(1-q)^{2}} = p\frac{1}{p^{2}} = \frac{1}{p}$
\vspace{5mm}

* $E(x(x-1)) = E(x^{2})-E(x) \Longrightarrow E(x^{2}) = E(x(x-1)) + E(x)$
\vspace{5mm}

$\Longrightarrow E(x(x-1)) = \sum_{x = 1}^{\infty} x(x-1)pq^{x-1} = pq\sum_{x = 1}^{\infty} x(x-1)q^{x-2}$ 
\vspace{2mm}

$= pq\sum_{x = 1}^{\infty} \frac{\partial^{2}}{\partial q^{2}} (q^{x}) = pq\frac{\partial^{2}}{\partial q^{2}} (\sum_{x = 1}^{\infty} q^{x}) = pq\frac{\partial^{2}}{\partial q^{2}} (\frac{q}{1-q}) = pq\frac{2}{(1-q)^{3}}$ 
\vspace{2mm}

$= pq\frac{2}{p^{3}} = q\frac{2}{p^{2}}$ 
\vspace{5mm}

$\Longrightarrow E(x^{2}) = \frac{2q}{p^{2}} + \frac{1}{p} = \frac{2q+p}{p^{2}} = \frac{2-p}{p^{2}}$
\vspace{4mm}

$\Longrightarrow Var(x) = E(x^{2}) - E(x)^{2} = \frac{2-p}{p^{2}} - \frac{1}{p^{2}} = \frac{1-p}{p^{2}} = \frac{1-0.2}{0.2^{2}} = \frac{0.8}{0.04} = 20$
\vspace{4mm}

$\Longrightarrow P(\abs{\bar{X}-\mu}<0.1Var(x)) =  P(\frac{\abs{\bar{X}-\mu}}{\sigma/\sqrt{n}}<\frac{0.1(20)}{\sqrt{20}/\sqrt{40}}) = P(\abs{Z}<2.83)$
\vspace{3mm}

$= P(-2.83<Z<2.83) = P(Z<2.83)-P(Z<-2.83)$
\vspace{3mm}

$= 0.99767-0.00233 = \fcolorbox{blue}{white}{0.99534}$
\vspace{10mm}

\textcolor{red}{10)} El tiempo de vida de una bater'ia es una variable aleatoria $X$ con distribuci'on exponencial de par'ametro: $\frac{1}{\theta}$. Se escoge una muestra de $n$ bater'ias.
\vspace{0.25mm}

a) Halle el error est'andar de la media muestral $\bar{X}$.

b) Si la muestra aleatoria es de tamaño $n = 64$, ¿con qu'e probabilidad diferir'a $\bar{X}$ del valor verdadero de $\theta$ en menos de un error est'andar?

c) ¿Qu'e tamaño de muestra m'inimo ser'ia necesario para que la media muestral $\bar{X}$ tenga un error est'andar menor a un $5\%$ del valor real de $\theta$?

d) Asumiendo muestra grande, qu'e tamaño de muestra ser'ia necesario para que $\bar{X}$ difiera de $\theta$ en menos del $10\%$ de $\theta$ con $95\%$ de probabilidad.
\vspace{4mm}

\textcolor{red}{Soluci'on:}
\vspace{4mm}

$f(x) = \frac{1}{\theta}e^{\frac{-x}{\theta}},  0\leq x \geq\infty$
\vspace{4mm}

$\Longrightarrow E(x) = \int_0^\infty \frac{x}{\theta}e^{\frac{-x}{\theta}}dx = \int_0^\infty xe^{\frac{-x}{\theta}}d(\frac{x}{\theta}) = \theta\int_0^\infty \frac{x}{\theta}e^{\frac{-x}{\theta}}d(\frac{x}{\theta}) = \theta\Gamma(1) = \theta$
\vspace{5mm}

$\Longrightarrow Var(x) = \int_0^\infty \frac{x^{2}}{\theta}e^{\frac{-x}{\theta}}dx - E(x)^{2}= \int_0^\infty x^{2}e^{\frac{-x}{\theta}}d(\frac{x}{\theta}) - \theta^{2}$
\vspace{3mm}

$= \theta^{2}\int_0^\infty (\frac{x}{\theta})^{2}e^{\frac{-x}{\theta}}d(\frac{x}{\theta}) - \theta^{2} = \theta^{2}\Gamma(2) - \theta^{2} = 2\theta^{2} - \theta^{2} =  \theta^{2}$
\vspace{7mm}

a) 
$E.S. = \sqrt{\frac{\sigma^{2}}{n}} = $\fcolorbox{blue}{white}{$\frac{\theta}{\sqrt{n}}$}
\vspace{5mm}

b)
$P(\abs{\bar{X}-\mu}<\frac{\theta}{\sqrt{n}}) =  P(\frac{\abs{\bar{X}-\mu}}{\sigma/\sqrt{n}}<\frac{\theta/\sqrt{n}}{\theta/\sqrt{n}}) = P(\abs{Z}<1) = P(-1<Z<1)$
\vspace{1mm}

$= P(Z<1)-P(Z<-1) = 0.84134-0.15866 = \fcolorbox{blue}{white}{0.6826}$
\vspace{7mm}

c)
$\frac{\theta}{\sqrt{n}}<0.05\theta \Longrightarrow \frac{1}{\sqrt{n}}<0.05 \Longrightarrow \frac{1}{0.05}<\sqrt{n} \Longrightarrow 20^{2}<\sqrt{n}^{2} $
\vspace{3mm}

$\Longrightarrow 20<\sqrt{n} \Longrightarrow \fcolorbox{blue}{white}{400<n}$
\vspace{6mm}

d)
$P(\abs{\bar{X}-\mu}<0.1\theta) = 0.95 \Longrightarrow P(\frac{\abs{\bar{X}-\mu}}{\sigma/\sqrt{n}}<\frac{0.1\theta}{\theta/\sqrt{n}}) = 0.95 $
\vspace{3mm}

$\Longrightarrow P(\abs{Z}<0.1\sqrt{n}) = 0.95 \Longrightarrow P(-0.1\sqrt{n}<Z<0.1\sqrt{n}) = 0.95 $
\vspace{3mm}

$\Longrightarrow P(Z<0.1\sqrt{n})-P(Z<-0.1\sqrt{n}) = 0.95 $
\vspace{3mm}

$\Longrightarrow P(Z<0.1\sqrt{n})-(1 -P(Z<0.1\sqrt{n}) = 0.95 $
\vspace{3mm}

$\Longrightarrow 2.P(Z<0.1\sqrt{n}) = 1.95 \Longrightarrow P(Z<0.1\sqrt{n}) = 0.975 $
\vspace{3mm}

$\Longrightarrow 0.1\sqrt{n} = 1.96 \Longrightarrow 19.6 = \sqrt{n}\Longrightarrow \fcolorbox{blue}{white}{n = 385}$

\end{document}
