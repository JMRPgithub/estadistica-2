\documentclass[12pt,a4paper,openany]{article}
\usepackage[utf8]{inputenc}
\usepackage{amsmath}
\usepackage{amsfonts}
\usepackage{amssymb}
\usepackage{graphicx}
\usepackage{enumerate}
\usepackage{multirow} 
\usepackage[width=0.00cm, left=3.00cm, right=2.50cm, top=3.00cm, bottom=2.50cm]{geometry}
\begin{document}
	\begin{enumerate}[6]
		\item una empresa comercializadora de cafe sabe que el consumo mensual(kg) de cafe por casa esta normalmente distribuida con una media desconocida u y una variacion estandar de 0.30. si se toma una muestra aleatoria de 36 casas y se registra su consumo de cafe durante un mes. ¿ cual es la probabilidad de que la media de ka muestra este entre los valores de u-0.1 y u+0.1?
	\end{enumerate}}
	 sol:\\
	x: consumo mesual en kg\\
	u\\
	var(x) = 0.30\\
	n = 36\\
	P (u-0.1 \le x \le u+0.1) = P( x \le u+0.1)- P( x \le u-0.1)\\
	=P( x-u/o/sqrt(n) \le u+0.1-u/)- P( x-u/o/sqrt(n) \le u-0.1-u)\\
	=P( z \le 2)- P( z \le -2)\\
	=0.97725-0.02275
		
		
\end{document}