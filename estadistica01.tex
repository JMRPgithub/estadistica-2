\documentclass[12pt, a4paper]{article}
\usepackage[T1]{fontenc}
\usepackage[latin1]{inputenc}
\usepackage{amsmath}
\usepackage[spanish]{babel}
\parindent = 0cm

\begin{document}

\textbf{Estudiante: }Luis Gerardo Dalguerre Garc�a\\
\textbf{C�digo: }27180119\\


11. La utilidad por la venta de cierto art�culo, en miles de soles, es una variable aleatoria con distribuci�n normal. En el 5\% de las ventas, la utilidad ha sido menor que 3.42, mientras que el 1\% de las ventas ha sido mayor que 19.32. Si se realizan 16 operaciones de ventas, �cu�l es la probabilidad de que el promedio de la utilidad por cada operaci�n est� entre \$10,000 y \$12,000?

\begin{itemize}
\item De acuerdo a los datos:
\begin{itemize}
\item $P(X < 3.42) = 0.05$
\item $P(X > 19.32) = 0.01$
\item $n = 16$
\end{itemize} 
\item Se estandariza y se igualan las ecuaciones para hallar la media $\mu$ y la varianza $\sigma$:
\begin{itemize}
\item {
$P(\dfrac{X - \mu}{\sigma} < \dfrac{3.42 - \mu}{\sigma}) = 0.05$\\[2mm]
$P(Z < \dfrac{3.42 - \mu}{\sigma}) = 0.05$\\[2mm]
Mediante la tabla:\\[3mm]
\begin{equation}\label{ec1}
Z = -1.64 = \dfrac{3.42 - \mu}{\sigma}
\end{equation}
}
\item {
$P(\dfrac{X - \mu}{\sigma} > \dfrac{19.32 - \mu}{\sigma}) = 0.01$\\[2mm]
$1 - P(Z \leq \dfrac{19.32 - \mu}{\sigma}) = 0.01$\\[2mm]
$P(Z \leq \dfrac{19.32 - \mu}{\sigma}) = 0.99$\\[2mm]
Mediante la tabla:\\[3mm]
\begin{equation}\label{ec2}
Z = 2.33 = \dfrac{19.32 - \mu}{\sigma}
\end{equation}
}
\newpage
\item {
Despejando de las ecuaciones \eqref{ec1} y \eqref{ec2} el t�rmino $\frac{\mu}{\sigma}$ e igualando las ecuaciones resultantes:\par
\[\frac{3.42}{\sigma} + 1.64 = \frac{19.32}{\sigma} - 2.33\]
\[3.97 = \frac{19.32 - 3.42}{\sigma}\]
\[\sigma = 4\]
}
\item {
Se obtiene la media:\\
\[\frac{19.32 - \mu}{4} = 2.33\]
\[\mu = 10\]
}
\item {
Sea $X_{m}$ la media muestral y dividiendo 10,000 y 12,000 entre 1000:
\begin{align*}
&= P(10 \leq X_{m} \leq 12)\\[2mm]
&= P(\dfrac{10 - 10}{\dfrac{4}{4}} \leq \dfrac{X_{m} - \mu}{\dfrac{\sigma}{\sqrt{n}}} \leq \dfrac{12 - 10}{\dfrac{4}{4}})\\[2mm]
&= P(0 \leq Z \leq 2)\\[2mm]
&= P(Z \leq 2) - P(Z \leq 0)\\[2mm]
&= 0.47725
\end{align*}
}
\end{itemize} 
\end{itemize}


12. La vida �til de cierta marca de llantas radiales es una variable aleatoria \textit{X} cuya distribuci�n es normal con $\mu$ = 38,000 km y $\sigma$ = 3,000 km.\par
\hspace{0.6cm} a) Si la utilidad \textit{Y} (en \$) que produce cada llanta est� dada por la relaci�n: $Y = 0.2X + 100$, �cu�l es la probabilidad de que la utilidad sea mayor que 8,900\$?\par
\hspace{0.6cm} b) Determine el n�mero de tales llantas que debe adquirir una empresa de transporte para conseguir una utilidad media de al menos 7541\$ con probabilidad 0.996.\par

\begin{itemize}
\item De acuerdo a lo datos, se halla $E(Y)$, $Var(Y)$ y la $P(Y > 8900)$:
\begin{itemize}
\item {
Se halla la media:
\begin{align*}
E(Y) &= 0.2E(X) + E(100)\\[2mm]
E(Y) &= 0.2(38,000) + 100\\[2mm]
E(Y) &= 7700 = \mu
\end{align*}
}
\item {
Se halla la varianza:
\begin{align*}
Var(Y) &= 0.2^{2}Var(Y) + Var(100)\\[2mm]
Var(Y) &= 0.04(3000)^{2} + 0\\[2mm]
Var(Y) &= 360,000 = \sigma^{2}
\end{align*}
}
\item {
Entonces, la probabilidad:
\begin{align*}
&= P(Y > 8900)\\[2mm]
&= 1 - P(Y \leq 8900)\\[2mm]
&= 1 - P(\dfrac{Y - \mu}{\sigma} \leq \dfrac{8900 - 7700}{600})\\[2mm]
&= 1 - P(Z \leq 2)\\[2mm]
&= 0.0228
\end{align*}
}
\end{itemize}
\item Sea $Y_{m}$ la utilidad media, entonces se halla \textit{n}:
\begin{align*}
P(Y_{m} \geq 7541) &= 0.996\\[2mm]
1 - P(Y_{m} < 7541) &= 0.996\\[2mm]
P(\dfrac{Y_{m} - \mu}{\dfrac{\sigma}{n}} < \dfrac{7541 - 7700}{\dfrac{600}{\sqrt{n}}}) &= 0.004\\[2mm]
P(Z < \dfrac{(7541 - 7700)\sqrt{n}}{600}) &= 0.004
\end{align*}
Mediante la tabla:\\
\begin{align*}
Z = -2.65 &= \dfrac{(7541 - 7700)\sqrt{n}}{600}\\[2mm]
\dfrac{n(25281)}{360000} &= 7.0225\\[2mm]
n &= 100\\[2mm]
\end{align*}
\end{itemize}


13. Un proceso autom�tico llena de bolsas de caf� cuyo peso neto tiene una media de 250 gramos y una desviaci�n est�ndar de 3 gramos. Para controlar el proceso, cada hora se pesan 36 bolsas escogidas al azar; si el peso neto medio est� entre 249 y 251 gramos se contin�a con el proceso aceptando que el peso neto medio es 250 gramos y en caso contrario, se detiene el proceso para reajustar la m�quina.\par
\hspace{0.6cm} a) �Cu�l es la probabilidad de detener el proceso cuando el peso neto medio realmente es 250?\par
\hspace{0.6cm} b) �Cu�l es la probabilidad de aceptar que el peso neto promedio es 250 cuando realmente es de 248 gramos?\par

\begin{itemize}
\item Pregunta a):
\begin{align*}
&= P(X_{m} < 249) + P(X_{m} > 251)\\[2mm]
&= P(Z < \dfrac{249 - 250}{\dfrac{3}{\sqrt{36}}}) + [1 - P(Z \leq \dfrac{251 - 250}{\dfrac{3}{\sqrt{36}}})]\\[2mm]
&= P(Z < -2) + [1 - P(Z \leq 2)]\\[2mm]
&= 0.0228 + 0.0228\\[2mm]
&= 0.0456
\end{align*}
\item Pregunta b):
\begin{align*}
&= P(X_{m} < 249)\\[2mm]
&= P(Z < \dfrac{249 - 250}{\dfrac{3}{\sqrt{36}}})\\[2mm]
&= P(Z < -2)\\[2mm]
&= 0.0228
\end{align*}
\end{itemize}

\end{document}