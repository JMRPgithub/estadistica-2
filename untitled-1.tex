\documentclass{article}
\usepackage{inputenc}

\title{Trabajo estadistica}
 \author{ 
 Anccasi Garcia, Roy\\ 
 \texttt{27182301} \\ 
 \texttt{$correo@electronico1$} 
 \and 
 sujeto 1\\ 
 \texttt{informacionRelevante} \\ 
 \texttt{$correo@electronico2$} 
\and 
 sujeto 2\\ 
 \texttt{informacionRelevante} \\ 
 \texttt{$correo@electronico2$} 
\and 
 sujeto 3\\ 
 \texttt{informacionRelevante} \\ 
 \texttt{$correo@electronico2$} 
\and 
 sujeto 4\\ 
 \texttt{informacionRelevante} \\ 
 \texttt{$correo@electronico2$} 
\and 
sujeto 5\\ 
 \texttt{informacionRelevante} \\ 
 \texttt{$correo@electronico2$} 
} 

\usepackage{amsmath}
\usepackage{amssymb}



\begin{document}
\maketitle
1. Un taller tiene 5 empleados. Los salarios diarios en dolares de cada uno de ellos seon: 5, 7, 8, 10, 10\\
a) Determine la media y la varianza de la poblacion\\
\begin{table}[htb]
\centering
\begin{tabular}{| p{2.2cm}| p{2.2cm} |}
\hline
\multicolumn{2}{|c|}{Salarios} \\
\hline
$x_{i}$ & valor\\\hline
$x_{1}$ & 5 \\ \hline
$x_{2}$ & 7 \\ \hline
$x_{3}$ & 8 \\ \hline
$x_{4}$ & 10 \\ \hline
$x_{5}$ & 10 \\ \hline
\end{tabular}
\caption{Tabla de salarios.}
\end{table}

Hallamos la media
$$\bar{x}=  \displaystyle\sum_{i=1}^n\frac{x_{i}}{n}=\frac{x_{1}+x_{2}+x_{3}+x_{4}+x_{5}}{n}=\frac{5+7+8+10+10}{5}=\frac{40}{5}$$
tenemos que , $\bar{x}=8$\\
   Para la varianza
$$\sigma^2=\displaystyle\sum_{i=1}^n\frac{(x_{i}-\bar{x})^2 x}{n}=\frac{(x_{1}-\bar{x})^2+(x_{2}-\bar{x})^2+(x_{3}-\bar{x})^2+(x_{4}-\bar{x})^2+(x_{5}-\bar{x})^2}{n}$$ 
$$=\frac{(5-8)^2+(7-8)^2+(8-8)^2+(10-8)^2+(10-8)^2}{5}=\frac{9+1+0+4+4}{5}=\frac{18}{5}=3.6$$
tenemos que , $\sigma^2=3.6$

2. De la historia sacada de los registros de la Universidad se ha determinado que las calificaciones de MATE I y de FILO I se distribuyen normalmente con las medias respectivas de 12 y 15 y con varianzas homogeneas igual a 4 . ¿Cual es la probabilidad de que la media de las notas de un alumno que llevo tales cursos este entre 13 y 16?\\


3.Si $\bar{x}$ denota la media de la muestra aleatoria $x_{1},x_{2},..,x_{9}$ de tamano 9 escogida de la poblacion (X) normal $N(6,6^2)$.\\
a)Describa la deistribucion de probabilidades de la variable $ \bar{X}$\\
b)Halle el percentil 80 de la distribucion de  $ \bar{X}$\\
$$P( \frac{\bar{x}-\mu}{\frac{\sigma}{\sqrt{n}}}\leq \frac{k-\mu}{\frac{\sigma}{\sqrt{n}}} )=0.8$$
$$P( \frac{\bar{x}-\mu}{\frac{\sigma}{\sqrt{n}}}\leq \frac{k-6}{\frac{6}{\sqrt{9}}} )=0.8$$
teniendo, $$ \frac{k-6}{\frac{6}{\sqrt{9}}} =0.84$$
$$ \frac{k-6}{2} =0.84$$
$$ k-6 =1,68$$, $$ k =7,68$$
c)Si Y=3X-5, calcular P[$\bar{Y}>28$].\\
entonces, $$P[\bar{Y}>28]=P[3\bar{X}-5>28]=P[3\bar{X}>33]=P[\bar{X}>11]=1-P[\bar{X}\leq11]$$
$$=1-P( \frac{\bar{x}-\mu}{\frac{\sigma}{\sqrt{n}}}\leq \frac{11-6}{\frac{6}{\sqrt{9}}} )=1-P( Z\leq 2,5 )=1-0.9938=0.0062$$

4. Una compania agroindustrial ha logrado establcer el siguiente modelo de probabilidad discreta de los sueldos(X) en cientos de dolares de su personal:

\begin{table}[htb]
\centering
\begin{tabular}{|l|l|l|l|l|l|}
\hline
         x & 1 &2 & 3& 4& 5  \\ \cline{1-6}
$f(x)=P[X=x]$& 0.1 &0.2 & 0.4&0.2&0.1\\ \cline{1-6}

\end{tabular}
\caption{Sueldos en cientos de dolares.}
\label{tabla:final}
\end{table}
Si de esta pobalcion de sueldos se toman 30 sueldos al azar.\\
a) Halle la media y la varianza de la media muestral.
\begin{table}[htb]
\centering
\begin{tabular}{| p{2.2cm}| p{2.2cm} | p{2.2cm} | p{2.2cm} |}
\hline
$x_{i}$ & f(x)& xf(x)& $x^2f(x)$\\ \hline
1 & 0.1 & 0.1&  0.1\\ \hline
2 & 0.2 & 0.4 & 0.8\\ \hline
3 & 0.4 & 1.2 & 3.6\\ \hline
4 & 0.2 & 0.8& 3.2\\ \hline
5 & 0.1 & 0.5& 2.5\\ \hline
\end{tabular}
\caption{xf(x).}
\end{table}

Entonces para calcular $\mu$, $$\lambda=\displaystyle\sum_{i=1}^nx_{i}f(x_{1})=0.1+0.4+1.2+0.8+0.5=3$$
Para la varianza, $$\sigma^2_{x}=\displaystyle\sum_{i=1}^n(x_{i})^2-\mu^2=0.1+0.8+3.6+3.2+2.5-9= 10.2-9=1.2$$\\
entonces,  $$\sigma^2_{\bar{x}}=\frac{\sigma^2_{x}}{n}=\frac{1,2}{30}=0.04$$

b) Calcule la probabilidad de que la media muestral este entre 260 y 330 dolares.//
Recoredemos que los sueldos estan en base 100, entonces debemos hallar, $P(2.6\leq  \bar{x}\leq  3.3)$//
$$P(2.6\leq  \bar{x}\leq  3.3)=P( \bar{x}\leq  3.3 )-P(\bar{x}\leq2.6  )$$
$$=P( \frac{\bar{x}-\mu}{\frac{\sigma}{\sqrt{n}}}\leq  frac{3.3-3}{0.2} )-P( \frac{\bar{x}-\mu}{\frac{\sigma}{\sqrt{n}}}\leq\frac{2.6-3}{0.2} )=P(Z\leq 1.5)-P(Z\leq -2)$$
$$=0.9332-0.0183 =0.9149$$


\end{document}
