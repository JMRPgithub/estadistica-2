%\documentclass[12pt,a4paper]{report}
\documentclass[DIV=calc,paper=a4,fontsize=11pt,openany]{book}
\usepackage[utf8]{inputenc}
\usepackage[spanish]{babel}
\renewcommand{\baselinestretch}{1.3}
\usepackage[autostyle,spanish=mexican]{csquotes}

\usepackage{amsmath,amsfonts,amsthm,amssymb,wasysym}
\usepackage{graphicx}
\usepackage{amsmath}
\usepackage{amsfonts}
\usepackage{latexsym}
\usepackage{enumerate}
\usepackage{amssymb}
\usepackage{makeidx}
\usepackage{tikz}
\usepackage{pstricks} 
\usepackage{cancel}
\usepackage{caption}
\usepackage{float} %flotador de label
\usepackage{wrapfig} %Permite figuras o tablas que tienen texto
\usepackage{shadow}
\usepackage{fancyhdr} %definir pie de pagina y encabesados
\usepackage{multicol}
\usepackage[all]{xy}
\usepackage{underoverlap}
%\usepackage{asymptote}º
\usepackage{environ}
\usepackage{multicol}
\usepgflibrary{shapes.misc}
%\usepackage{MnSymbol}
\usepackage{mathrsfs}
\usepackage[left=3cm,right=3.7cm,top=2.8cm,bottom=2.7cm]{geometry}
\usepackage{pdfpages}
\usepackage[colorinlistoftodos, textwidth=3cm, shadow]{todonotes}
 \definecolor{verdeoscuro}{RGB}{6,18,97}
%%%encabezado y pie de pagina1
\renewcommand{\footrulewidth}{2pt}  %regla de pie de pagina
\pagestyle{fancy} %para que aparescaa regla orizontal
\columnseprule=1.2pt %regla de columna
\columnsep=25pt  %espacio en blando de la columna y texto
%encabezado
%\lhead[{\vspace*{-0.2cm}UNIVERSIDAD NACIONAL S.C.H} ]{\bf\red CÁLCULO II}
\chead[]{}
\rhead[\rightmark]{\bf \blue }
%pie de pagina
\definecolor{blue(pigment)}{RGB}{20,70,60}
\lfoot[]{\bf }
\cfoot[]{}
\rfoot[]{\vspace*{-0.8cm}\fcolorbox{black}{white}{\Large\bfseries\color{blue(pigment)} {\thepage}}}
\usetikzlibrary{shadows}                          %
\newcommand{\preg}[1]{
   \par\vspace{0.6em}\noindent                         %
\begin{tikzpicture}[baseline=(X.base)]\node       %
[drop shadow,fill=black,draw,very thin] (X) {\color{white} \textsf{\textbf{#1}}}; %
\end{tikzpicture}\hspace{0.4cm}} 
% ejemplo                          %
\newcommand{\sombra}[1]{                          %
\begin{tikzpicture}[baseline=(X.base)]\node       %
[drop shadow,fill=white,draw,very thin] (X) {#1}; %
\end{tikzpicture}} 
\newcommand{\ejemplo}[1]{
% \par\vspace{0.6em}\noindent   
\begin{tikzpicture}
\node[rounded rectangle, white, fill=black!70, draw]
{\sf {Ejemplo} {\normalsize  #1}};
\end{tikzpicture}
}
                                %
%%                    %                           %
%%%%%%%%%%%%%%%%%%%%%%%%%%%%%%%%%%%%%%%%%%%%%%%%%%%%%%%%%%%%%%%%%%%%%%%%%
\renewcommand{\thesection}{\arabic{section}}
%%%%%%%contador%%%%%%%%%%%%%%%%%
\newcounter{contador}[section] %
\setcounter{contador}{0}       %
%%%%%%%%%%%%%%%%%%%%%%%%%%%%%%%%
%%%%%%%%%%%%%%%%%%%%%%%%%%%%%%%%%%%%%%%%%%%%%%%%%%%%%%%%%%
%%%%%%%%definimos ambiente pregunta y solucion%%%%%%%%%%%%
\usepackage{colortbl}                                    %
%\makeatletter                                            %
\definecolor{col}{RGB}{45,07,107}
                    %
\newcommand\sol{\noindent                                         %
 \\[0.4em]
  { \slshape \color{red}\textbf{Resolución}}%
  \\[0.3em]\noindent  }
                     %
\makeatother
                                %
%%%%%%%%%%%%%%%%%%%%%%%%%%%%%%%%%%%%%%%%%%%%%%%%%%%%%%%%%%
%%%%%%%%%%%%%%%%%%%%%%%%%%%%%%%%%%%%%%%%%%%%%%%%%%%%%%%%%%
%%%%comandos particulares%%%%%%%%%%%%%%%%%%%%%%%%%%%%%%%%%%%%%%
\newcommand{\af}{\alpha}
\newcommand{\D}{\pmb{D}}
\newcommand{\be}{\beta}
\newcommand{\te}{\theta}
\newcommand{\al}{\displaystyle}
\newcommand{\med}{\scriptstyle}
\newcommand{\imp}{\Rightarrow}
\newcommand{\impp}{\Longleftrightarrow}
\newcommand{\im}{\rightarrow}
\newcommand{\p}{\cdot}
\newcommand{\ca}{\cancel}
\newcommand{\tri}{\therefore}
\newcommand{\sen}{\mathop{\rm sen}\nolimits}
\newcommand{\tg}{\mathop{\rm tg}\nolimits}\newcommand{\ct}{\mathop{\rm ctg}\nolimits}
\newcommand{\R}{\mathbb{R}}
\newcommand{\Z}{\mathbb{Z}}
\newcommand{\N}{\mathbb{N}}
\newcommand{\se}{\overline}
\newcommand{\ang}{\measuredangle}
\newcommand{\an}{\angle}
\newcommand{\tr}{\triangle}
\newcommand{\g}{^\circ}
\newcommand{\spa}{\smallskip}
\newcommand{\spac}{\medskip}
\newcommand{\brazo}{\underbrace}
\newcommand{\medi}{\mbox{m}\measuredangle}
\newcommand{\f}{\frac}
\newcommand{\dist}{\vspace{-0.5cm}}
\newcommand{\rad}{\;\mbox{rad}}\newcommand{\bi}{\bullet\;\;}
\newcommand{\Del}[1]{\left[#1\right]}
\newcommand{\del}[1]{\left(#1\right)}
\newcommand{\sect}[1]{\leftslice\scriptscriptstyle{#1}}
\newcommand{\q}{\sqrt}
\newcommand{\y}{\;\wedge\;}
\newcommand{\oo}{\;\vee \;}
\newcommand{\sii}{\;\Leftrightarrow\;}
\newcommand{\pt}{\times}
\newcommand{\s}{\sec\mathsf{h}}
\newcommand{\ta }{\tan\mathsf{h}}
\newcommand{\limt}{\lim_{n\rightarrow \infty}}
\newcommand{\prt}{\partial}
\newcommand{\M}{\geq}
\newcommand{\m}{\leq}
\newcommand{\vol}{\mathrm{Vol}}
\newcommand{\area}{\mathrm{Area}}
\newcommand{\yy}{\overline{y}}
\newcommand{\vx}{\overline{x}}
\newcommand{\hs}{\hspace{0.6cm}}
\newcommand{\dd}{\mathrm{d}}
\newcommand{\n}{\nabla}
%\newcommand{\ejemplo}[1]{\color{white}\textsf{Ejemplo \yellow #1}}                                 
\usepackage{amsmath}
\usepackage{graphicx,tcolorbox}
\definecolor{blue(pigment)}{RGB}{20,70,60}
\definecolor{som}{RGB}{68,05,5}
\usetikzlibrary{shadows} 
%---------RTA-------------
\newcommand{\ptas}[1]{
\begin{flushright}
\begin{tikzpicture}
\node[rounded rectangle, white, fill=black!70, draw]
{\sf {clave} {\normalsize #1}};
\end{tikzpicture}
\end{flushright}}

% Estilos: Sonny, Lenny, Glenn, Conny, Rejne, Bjarne, Bjornstrup
%\usepackage[Sonny]{fncychap}
\begin{document} 
%%%portada%%%%%%%%%%%%
\thispagestyle{empty}

   % Sangría
%%\input{caratula/cara}
%%%%%%%%%%%%%%%%%%
\newpage




\begin{center}
\large{\textcolor{black}{\textbf{\underline{\textit{EJERCICIOS DE ESTADÍSTICA II}}}}}
\end{center}

\textcolor{blue}{HUAMAN LANDA,Leonel Código: 27180701}\\

\textcolor{red}{Ejercicio 17}\\

Una empresa vende bloques de mármol cuyo peso se distribulle normalmente con una media de 200 kilogramos.

a) Calcule la varianza del peso de los bloques, si la probabilidad de que el peso este entre 165 Kg y 235 Kg es 0.9876.\\

b) ¿Qué tan grande debe ser la muestra para que haya una probabilidad de 0.9938 de que el peso medio de la muestra sea inferior a 205 Kg?\\

\textcolor{red}{Solución: a)}

$\mu = 200 Kilogramos$\\
$\sigma = ?$\\
$\textit{P}(165 < \overline{X} < 235) = 0.9876$\\

$\textit{P}(165 < \overline{X} < 235) = \textit{P}(\dfrac{165 - \mu}{\sigma} < \dfrac{\overline{X} - \mu}{\sigma} < \dfrac{235 - \mu}{\sigma})$\\

$\textit{P}(165 < \overline{X} < 235) = \textit{P}(\dfrac{165 - 200}{\sigma} < \textit{Z} < \dfrac{235 - 200}{\sigma})$\\

$\textit{P}(165 < \overline{X} < 235) = \textit{P}(\dfrac{-35}{\sigma} < \textit{Z} < \dfrac{35}{\sigma})$\\

$\textit{P}(165 < \overline{X} < 235) = \textit{P}(\textit{Z} < \dfrac{35}{\sigma}) - \textit{P}(\textit{Z} < \dfrac{-35}{\sigma})$\\

$\textit{P}(165 < \overline{X} < 235) = \textit{P}(\textit{Z} < \dfrac{35}{\sigma}) - (1 -\textit{P}(\textit{Z} < \dfrac{35}{\sigma}))$\\

$\textit{P}(165 < \overline{X} < 235) = 2\textit{P}(\textit{Z} < \dfrac{35}{\sigma}) - 1 $\\

\textcolor{blue}{Como: $\textit{P}(165 < \overline{X} < 235) = 0.9876$}\\

$0.9876 = 2\textit{P}(\textit{Z} < \dfrac{35}{\sigma}) - 1 $\\

$1.9876 = 2\textit{P}(\textit{Z} < \dfrac{35}{\sigma})$\\

$0.9938 = \textit{P}(\textit{Z} < \dfrac{35}{\sigma})$\\

\textcolor{blue}{$\textit{Z} = 2.50,  \quad F(\textit{Z})$}\\

$2.50 = \dfrac{35}{\sigma}$\\

\colorbox{yellow}{$\sigma = 14$}


\textcolor{red}{Solución: b)}\\

$\textit{n} = ?, \quad \sigma = 14 , \quad \mu = 200$\\

$\textit{P}(\overline{X} < 205) = 0.9938$\\

$\textit{P}(\overline{X} < 205) = \textit{P}(\dfrac{\overline{X} - \mu}{\sigma/\sqrt{\textit{n}}} < \dfrac{205 - \mu}{\sigma/\sqrt{\textit{n}}})$\\

$\textit{P}(\overline{X} < 205) = \textit{P}(\textit{Z} < \dfrac{205 - 200}{14/\sqrt{\textit{n}}})$\\

$\textit{P}(\overline{X} < 205) = \textit{P}(\textit{Z} < \dfrac{5\sqrt{\textit{n}}}{14})$\\

\textcolor{blue}{Como: $\textit{P}(\overline{X} < 205) = 0.9938$}\\

$0.9938 = \textit{P}(\textit{Z} < \dfrac{5\sqrt{\textit{n}}}{14})$\\

\textcolor{blue}{$\textit{Z} = 2.50, \quad F(\textit{Z})$}\\

$2.50 = \dfrac{5\sqrt{\textit{n}}}{14}$\\

$2.50 = \dfrac{5\sqrt{\textit{n}}}{14}$\\

$\sqrt{\textit{n}} = 7$

\colorbox{yellow}{$\textit{n} = 49$}\\






\textcolor{red}{Ejercicio 18}\\

La duración en horas de una marca de tarjeta electrónica  se distribulle exponencialmente con un promedio de 1000 horas.\\

a) Halle el tamaño \textit{n} de la muestra de manera que sea 0.9544 la probabilidad de que su media muestral este entre 800 y 1200 horas.\\

b) Si se obtiene una muestra aleatoria de 100 de esas tarjetas calcular la probabilidad que la duración media de la muestra sea superior a 1100 horas.\\


\textcolor{red}{Solución: a)}\\

$\mu = 1000 horas. \quad \textit{n} = ?$\\

$\textit{P}(800 < \overline{X} < 1200) = 0.9544$\\

Consideramos $\textit{E}(X) = \theta = 1000$\\

$\sigma^{2} = 1 000 000 \quad \sigma = 1000$\\

$\textit{P}(800 < \overline{X} < 1200)) = \textit{P}(\dfrac{800 - \mu}{\sigma/\sqrt{\textit{n}}} < \dfrac{\overline{X} - \mu}{\sigma/\sqrt{\textit{n}}} < \dfrac{1200 - \mu}{\sigma/\sqrt{\textit{n}}})$\\

$\textit{P}(800 < \overline{X} < 1200)) = \textit{P}(\dfrac{800 - 1000}{1000/\sqrt{\textit{n}}} < \dfrac{\overline{X} - \mu}{\sigma/\sqrt{\textit{n}}} < \dfrac{1200 - 1000}{1000/\sqrt{\textit{n}}})$\\

$\textit{P}(800 < \overline{X} < 1200) = \textit{P}(\dfrac{-200}{1000/\sqrt{\textit{n}}} < \textit{Z} < \dfrac{200}{1000/\sqrt{\textit{n}}})$\\

$\textit{P}(800 < \overline{X} < 1200) = \textit{P}(-0.2\sqrt{\textit{n}} < \textit{Z} < 0.2\sqrt{\textit{n}})$\\


$\textit{P}(800 < \overline{X} < 1200) = \textit{P}(\textit{Z} < 0.2\sqrt{\textit{n}}) - \textit{P}(\textit{Z} < -0.2\sqrt{\textit{n}})$\\

$\textit{P}(800 < \overline{X} < 1200) = \textit{P}(\textit{Z} < 0.2\sqrt{\textit{n}}) - (1 -\textit{P}(\textit{Z} < 0.2\sqrt{\textit{n}}))$\\

$\textit{P}(800 < \overline{X} < 1200) = 2\textit{P}(\textit{Z} < 0.2\sqrt{\textit{n}}) - 1 $\\

\textcolor{blue}{Como: $\textit{P}(800 < \overline{X} < 1200) = 0.9544$}\\

$0.9544 = 2\textit{P}(\textit{Z} < 0.2\sqrt{\textit{n}}) - 1 $\\

$1.9544 = 2\textit{P}(\textit{Z} < 0.2\sqrt{\textit{n}})$\\

$0.9772 = \textit{P}(\textit{Z} < 0.2\sqrt{\textit{n}})$\\

\textcolor{blue}{$\textit{Z} = 2, \quad F(\textit{Z})$}\\

$2 = 0.2\sqrt{\textit{n}}$\\

$\sqrt{\textit{n}} = 10$\\

\colorbox{yellow}{$\textit{n} = 100$}\\


\textcolor{red}{Solución: b)}\\

$\textit{n} = 100, \quad \mu = 1000, \quad \sigma = 1000 $\\

$\textit{P}(\overline{X} \geqslant 1100) = ?$\\

$\textit{P}(\overline{X} \geqslant 1100) = \textit{P}(\dfrac{\overline{X} - \mu}{\sigma/\sqrt{\textit{n}}} \geqslant \dfrac{1100 - \mu}{\sigma/\sqrt{\textit{n}}})$\\

$\textit{P}(\overline{X} \geqslant 1100) = \textit{P}(\textit{Z} \geqslant \dfrac{1100 - 1000}{1000/\sqrt{\textit{100}}})$\\

$\textit{P}(\overline{X} \geqslant 1100) = \textit{P}(\textit{Z} \geqslant \dfrac{100}{1000/10})$\\

$\textit{P}(\overline{X} \geqslant 1100) = \textit{P}(\textit{Z} \geqslant \dfrac{100}{100})$\\

$\textit{P}(\overline{X} \geqslant 1100) = \textit{P}(\textit{Z} \geqslant 1)$\\

$\textit{P}(\overline{X} \geqslant 1100) = 1 - \textit{P}(\textit{Z} \leqslant 1)$\\

\textcolor{blue}{Lectura de tabla. $\textit{P}(\textit{Z} \leqslant 1) = 0.8413$}\\

$\textit{P}(\overline{X} \geqslant 1100) = 1 - 0.8413$\\

\colorbox{yellow}{$\textit{P}(\overline{X} \geqslant 1100) = 0.1587$}\\


\textcolor{red}{Ejercicio 19}\\

Un procesador de alimentos envasa café en frascos de 400 gramos. Para controlar el proceso se selecionan 64 frascos cada hora. si su peso medio es inferior a un valor critico K, se detiene el proceso y se registra. En caso contrario, se continúa el operación sin detener el proceso. Determinar el valor de K de modo que haya una probabilidad de sólo 5\% de detener el proceso cuando está envasado a un promedio de 407.5 gramos con una desviación estándar de 2.5 gramos.\\


\textcolor{red}{Solución:}\\

$K = ?, \quad \textit{n} = 64, \quad \mu = 407.5, \quad \sigma = 2.5 $\\

$\textit{P}(\overline{X} \leqslant \textit{K}) = 0.05$\\

$\textit{P}(\overline{X} \leqslant \textit{K}) = \textit{P}(\dfrac{\overline{X} - \mu}{\sigma/\sqrt{\textit{n}}
} \leqslant \dfrac{K - \mu}{\sigma/\sqrt{\textit{n}}})$\\

$\textit{P}(\overline{X} \leqslant \textit{K}) = \textit{P}(Z \leqslant \dfrac{K - 407.5}{2.5/\sqrt{\textit{64}}})$\\

$\textit{P}(\overline{X} \leqslant \textit{K}) = \textit{P}(Z \leqslant \dfrac{K - 407.5}{0.3125})$\\


\textcolor{blue}{Como: $\textit{P}(\overline{X} \leqslant \textit{K}) = 0.05$}\\

$0.05 = \textit{P}(Z \leqslant \dfrac{K - 407.5}{0.3125})$\\

$\textit{P}(Z \leqslant \dfrac{K - 407.5}{0.3125}) = 0.05$\\

\textcolor{blue}{$\textit{Z} = -1.64 \quad F(\textit{Z})$}\\

$ \dfrac{K - 407.5}{0.3125} = -1.64$\\

$ K - 407.5 = -0.5125$\\

$ K  = 407.5 -0.5125$\\

\colorbox{yellow}{$K = 406.987$}


















\end{document}






